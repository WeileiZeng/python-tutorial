\def\rank{\mathop{\rm rank}}
\def\wgt{\mathop{\rm wgt}}
\def\lc{\mathop{\rm lc}}
\documentclass[aps,prb,12pt,tightenlines,%
notitlepage,longbibliography]{revtex4-1}

\usepackage{hyperref}
\usepackage{graphicx}
\usepackage{amsthm}
\usepackage{amsfonts}
\usepackage{array}
\usepackage{enumitem}
\newtheorem{theorem}{Theorem}
\newtheorem{statement}[theorem]{Theorem}
\newtheorem{note}[theorem]{Note}
\newtheorem{corollary}[theorem]{Corollary}
\newtheorem{conjecture}[theorem]{Conjecture}
\newtheorem{lemma}[theorem]{Lemma}
\newtheorem{definition}[theorem]{Definition}
%\advance\textheight by -3.2in
\begin{document}
\title{Lecture Note on Python}
\author{WZ}
\date{\today}
\begin{abstract}
  This note includes everything covered in the class. It will be
  updated every week.
\end{abstract}
\maketitle

\section{syllabus}
\textbf{Objectives}: In this class, you will learn the basics of
Python. You will be able to write excutable codes on your own and read
most python codes. Most importantly, you will learn how to treat and
solve problem in the sense of computer science.


\textbf{Instructor}: WZ

\textbf{Class Meetings}:

Sunday 2:30-4:30pm; Monday 4:30-6:30pm. Mar 31-Jun 10

20 lessons in 10 weeks, 2 hours for each lesson. In each lesson, the
first half will focus on the lecture, and the second half for practice
and tutoring.

This is an exercise-based class. The amount of knowledge you learned
will be proportional to the amount of lines you wrote in Python. There
will be assignment everyweek. We will have time to discuss or even do
it in class. There will also be a project, to be started
on week 5, finished on week 8, and revised in the last two weeks. 

\textbf{Textbooks}:

For the lectures, we will follow these
\href{http://uwpce-pythoncert.github.io/IntroPython2016a/session01.html}{Slides}.

For reference, this is a wikibook of python: 
\href{https://pymbook.readthedocs.io/en/latest/index.html}{Python for
  you and me}.

Sample codes, assignments, and other material will be updated on 
\href{https://github.com/WeileiZeng/python-tutorial}{GitHub}.

The most powerful tool I have ever used is \href{https://www.google.com/}{Google}.

\textbf{Prerequisite}

We can start learning programming at any level, as long as you have
stable access to a laptop for practicing.


\textbf{Tentative content}:

The following topics covers the basics in Python.
As everyone has
different needs when learning programming language. We will discuss
how much of them will be covered and what to be added. Especially we
can come up with some projects as preparation for the summer camp.

\begin{enumerate}

\item  Introduction
\item   gitHub, Functions, Booleans and Modules
\item   Sequences, Iteration and String Formatting
\item   Dictionaries, Sets, and Files
\item   Exceptions, Testing, Comprehensions
\item   Advanced Argument Passing, Lambda -- functions as objects
\item   Object Oriented Programming
\item   More OO -- Properties, Special methods
\item  Iterators, Iterables, and Generators
\item   Decorators, Context Managers, Regular Expressions, and Wrap Up
  
\end{enumerate}

More topics to be determined
\begin{enumerate}
\item
  Graphics user interface(GUI).
\item
  Image/data processing

  \end{enumerate}


\section{Lecture notes}
\subsection{Lecture 1 Introduction}
Introduction to python

Course outline

Sample codes and projects

Environment setup (Installation, Editor, GitHub)

HelloWrold!

\section{benchmark problem set(to be updated)}
These problem ensure you to understand various concepts in
Python. These are the source of assignment.

\begin{enumerate}
\item hello world!
\item summation of integers from 1 to 100
\item Fibonacci sequence
\item contact book
\item simple calculator
\item Tower of Hanoi
\item Tetris  

\end{enumerate}
  

\bibliography{WeileiBibFile}

\end{document}

