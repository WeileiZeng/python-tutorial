\def\rank{\mathop{\rm rank}}
\def\wgt{\mathop{\rm wgt}}
\def\lc{\mathop{\rm lc}}
\newcommand{\code}[1]{\texttt{#1}}
\documentclass[aps,prb,12pt,tightenlines,%
notitlepage,longbibliography]{revtex4-1}

%left alighment of title
\usepackage{etoolbox}
\patchcmd{\section}
  {\centering}
  {\raggedright}
  {}
  {}
\patchcmd{\subsection}
  {\centering}
  {\raggedright}
  {}
  {}


\usepackage{hyperref}
\usepackage{graphicx}
\usepackage{amsthm}
\usepackage{amsfonts}
\usepackage{array}
\usepackage{enumitem}
\newtheorem{theorem}{Theorem}
\newtheorem{statement}[theorem]{Theorem}
\newtheorem{note}[theorem]{Note}
\newtheorem{corollary}[theorem]{Corollary}
\newtheorem{conjecture}[theorem]{Conjecture}
\newtheorem{lemma}[theorem]{Lemma}
\newtheorem{definition}[theorem]{Definition}
%\advance\textheight by -3.2in
\begin{document}
\title{Lecture Note on Python}
\author{WZ}
\date{\today}
\begin{abstract}
  This note includes everything covered in the class. It will be
  updated every week.
\end{abstract}
\maketitle

\section{syllabus}
\textbf{Objectives}: In this class, you will learn the basics of
Python. You will be able to write excutable codes on your own and read
most python codes. Most importantly, you will learn how to treat and
solve problem in the sense of computer science.


\textbf{Instructor}: WZ

\textbf{Class Meetings}:

regular meeting on Sunday 2:30-4:30pm;

to be scheduled for Monday 4:30-6:30pm.

Mar 31-Jun 10

20 lessons in 10 weeks, 2 hours for each lesson. In each lesson, the
first half will focus on the lecture, and the second half for practice
and tutoring.

This is an exercise-based class. The amount of knowledge you learned
will be proportional to the amount of lines you wrote in Python. There
will be assignment everyweek. We will have time to discuss or even do
it in class. There will also be a project, to be started
on week 5, finished on week 8, and revised in the last two weeks. 

\textbf{Textbooks}:

For the lectures, we will follow these
\href{http://uwpce-pythoncert.github.io/IntroPython2016a/session01.html}{Slides}.

For reference, this is a wikibook of python: 
\href{https://pymbook.readthedocs.io/en/latest/index.html}{Python for
  you and me}.

Sample codes, assignments, and other material will be updated on 
\href{https://github.com/WeileiZeng/python-tutorial}{GitHub}.

The most powerful tool I have ever used is \href{https://www.google.com/}{Google}.

\textbf{Prerequisite}

We can start learning programming at any level, as long as you have
stable access to a laptop for practicing.

\newpage
\textbf{Tentative content}:

The following topics covers the basics in Python.
As everyone has
different needs when learning programming language. We will discuss
how much of them will be covered and what to be added. Especially we
can come up with some projects as preparation for the summer camp.

\begin{enumerate}

\item  Introduction
\item   gitHub, Functions, Booleans and Modules
\item   Sequences, Iteration and String Formatting
\item   Dictionaries, Sets, and Files
\item   Exceptions, Testing, Comprehensions
\item   Advanced Argument Passing, Lambda -- functions as objects
\item   Object Oriented Programming
\item   More OO -- Properties, Special methods
\item  Iterators, Iterables, and Generators
\item   Decorators, Context Managers, Regular Expressions, and Wrap Up
  
\end{enumerate}

More topics to be determined
\begin{enumerate}
\item
  Graphics user interface(GUI).
\item
  Image/data processing
\item
  Internet communication, (TCP/IP)
\item
  Editors (emacs), window manager (tmux), typesetting format (latex,
  markdown)
\item
  Game engine (Unity)
\end{enumerate}

\newpage
\tableofcontents

\section{Lecture notes}

\subsection{Lecture 1, Session 1, Mar 31, Introduction}
Introduction to python

Course outline

Sample codes and projects

Environment setup (Installation, Editor, GitHub)

HelloWorld!

\subsection{Lecture 2, Session 1, Apr 1, 2-4pm}
start at assignment

finish on function

\subsection{Lecture 3, Session 1-2, Apr 7, 2-4pm}

review mac/linux,python,emacs
review data type, function

start on function (return, parameter)

if, list, for

finish before variable scope, local v.s. global

\textbf{exercise in class}:FizzBuzz, exercise 2.2,2.3

\textbf{home exercise}: finish 2.2,2.3, do 2.4, think about 2.5


%%%%%%%%%%%%%%%%%%%%%%%%%%%%%%%%%%%%%%%%%%%%%%%%%%%%%%%%%%
\section{exercise}
For each exercise, please create a single \code{.py} file, e.g. \code{session1\_1.py}. Please put
all your files in one or several folders.
\subsection{session 1}
\begin{enumerate}
  \item *
    create folder \texttt{firstFolder}, create file
    \texttt{first\_file.txt}, write your name and date into that file;
    Copy this file to \texttt{second\_file.txt}, and write
    ``helloworld'' into the second file.

    Hint: to copy a file in mac, use \code{cp [name of file1] [name of file2]}
  \item *
    create helloworld.py; run it in python and shell respectively
  \item *
    assign some value to $a$ and $b$, calculate $c=a+b$ and print the
    result in the form ``c=[value of c]''
  \item *
    create a file to check the priority of \code{+-*/=}
  \item *
    assign value of a,b,print them; switch value of them, and print
    the result

  \item
    prepare a conversation between Jack and Rose, 5 sentence each.
    
    (This material will be used in several late programs: (a) print
    the conversation; (b) enunciate the conversation, using \code{say}.)
\end{enumerate}

\subsection{session 2}
\begin{enumerate}
\item
  warmup exersice
\item
  (a) print sequence 1-100;
  (b) add sum of 1-100 and print the result
\item
  print sequence 100-1
\item
  (a) find average of three numbers
  (b) find maximum of three numbers
\item
  calculate prime numbers within 100
\item
  create random sequence of length 100
\item
  sort a sequence
\item
  calculate Pi
\item
  calculate square root of a number
\item
  game: 21 piont
\end{enumerate}

%%%%%%%%%%%%%%%%%%%%%%%%%%%%%%%%%%%%%%%%%%%%%%%%%%%%%%%%%%%%%%%%%%%%%%%%%%%%%%%%%%%%%
\subsection{ideas on big project}
\begin{enumerate}
\item
  a conversation robot. Giving questions and responds according to
  answer
\item
  conversation application between two computer
\item
  a graphics program?
\item
  flappy bird
\item
  interactive math test. give random question and count the rate and time.
\item
  2048
\end{enumerate}

\section{benchmark problem set(to be updated)}
These problem ensure you to understand various concepts in
Python. These are the source of assignment.

\begin{enumerate}
\item hello world!
\item summation of integers from 1 to 100
\item Fibonacci sequence
\item contact book
\item simple calculator
\item Tower of Hanoi
\item Tetris  

\end{enumerate}


\section{command list}
Limux/Mac terminal command
\begin{verbatim}
cd : enter a folder
ls : list of all the files in current folder
mkdir : create a new directly
rm : delete a file (be cautious!)
touch [file] : create file
less : view the file, press [q] to quit
cat : print the file
cp [file1] [file2] :copy file1 to file2
Tab : auto completement
man [cmd] : show the manual on how to use this command. for example
man cp
\end{verbatim}
editor: emacs, prefix Ctrl-x  or C-x

\begin{verbatim}
emacs + [filename] : open this file for editing
C-x C-s : save file
C-x C-c : exit
C-x C-f : find/open file
\end{verbatim}

how to run python in terminal
\begin{verbatim}
python : lauch python
python + [filename] : use python to excute this file
\end{verbatim}

python command
\begin{verbatim}
print() : print string in ()
exit()  or Ctrl-d: exit python
\end{verbatim}


\bibliography{WeileiBibFile}

\end{document}

